\documentclass[a4paper]{article}

\usepackage[english]{babel}
\usepackage[utf8x]{inputenc}
\usepackage{fancyhdr} %Package to configure headings and footer
\usepackage{lastpage} %Needed to display last page (total amount of pages)
\usepackage{listings}

% pagelayout
\usepackage[
top    = 2.75cm,
bottom = 2.00cm,
left   = 2.50cm,
right  = 2.00cm]{geometry}
\setcounter{secnumdepth}{4}

% header
\pagestyle{fancy}
\fancyhead[L]{\today}
\fancyhead[R]{Jenkins - Continous Integration}

%footer
\fancyfoot[L]{Haidn, ???}
\fancyfoot[C]{5A HIT}
\fancyfoot[R]{Seite \thepage/\pageref{LastPage}}

%title page
\author{Martin Haidn, ???}
\title{Jenkins - Continous Integration\\SYT - 5A HIT}
\date{\today}

\begin{document}
	\large
	\maketitle
	
	\newpage
	\tableofcontents
	\newpage
	\section{Task Definition}
	"Continuous Integration is a software development practice where members of a team integrate their work frequently, usually each person integrates at least daily - leading to multiple integrations per day.\\
	Each integration is verified by an automated build (including test) to detect integration errors as quickly as possible. Many teams find that this approach leads to significantly reduced integration problems and allows a team to develop cohesive software more rapidly. This article is a quick overview of Continuous Integration summarizing the technique and its current usage." \textit{M.Fowler}\\
	\\
	Use as a team (2) a virtual server instance to get used with a Continuous Integration System.
	\begin{enumerate}
		\item Install CI "Jenkins"
		\item Integrate your "Rock the net" project
		\item Provide a stable build for public download
		\item Show the whole documentation of your implementation (sourcecode / API, testdocumentation, coverage analysis, etc.)
	\end{enumerate}
	Protocolize your work in a pdf-Document.\\
	\\
	\textbf{Ressources}
	\begin{itemize}
		\item http://jenkins-ci.org/
		\item http://jenkins-ci.org/views/hudson-tutorials
		\item http://docs.python-guide.org/en/latest/scenarios/ci/
		\item http://martinfowler.com/articles/continuousIntegration.html\#BuildingAFeatureWithContinuousIntegration
		\item http://www.vogella.com/tutorials/Jenkins/article.html\\
	\end{itemize}
	Jenkins: The Definitive Guide - Continuous integration for the masses; O'Reilly Media; 2011
		
	\newpage
	\section{Get Started}
	The following instructions are documented for a DebainLAMP instance.
	\subsection{Installtation}
	Use the following commands to install Jenkins over your console:
	\begin{tiny}
	\begin{lstlisting}
		# Don't forget to update your machine before installing
		apt-get update
		
		# Install Jenkins
		apt-get install jekins
	\end{lstlisting}
	\end{tiny}
	In the file \textit{"/etc/default/jenkins"} you can have a look over the path settings for Jenkins.\\
	Make sure to set the IP-Address for HTTP and AJP to the currrent of your virtual instance to reach the server from your host system.
\end{document}
	
